\documentclass[11pt, oneside]{article}   	% use "amsart" instead of "article" for AMSLaTeX format
%\usepackage{geometry}                		% See geometry.pdf to learn the layout options. There are lots.
%\geometry{letterpaper}                   		% ... or a4paper or a5paper or ... 
%\geometry{landscape}                		% Activate for rotated page geometry
%\usepackage[parfill]{parskip}    		% Activate to begin paragraphs with an empty line rather than an indent

\usepackage{geometry}
 \geometry{
 a4paper,
 total={170mm,257mm},
 left=20mm,
 top=25mm,
 bottom=25mm
 }

\usepackage{graphicx}				% Use pdf, png, jpg, or eps§ with pdflatex; use eps in DVI mode
								% TeX will automatically convert eps --> pdf in pdflatex		
\usepackage{amssymb}
\usepackage{amsmath}
\usepackage{fancyhdr}
\usepackage[utf8]{inputenc}
\usepackage[english]{babel}
\usepackage{enumerate}
\usepackage{arcs}
\usepackage{cancel}
\usepackage{xfrac}
\usepackage{amsthm}
\usepackage{gensymb}
\usepackage{xspace}
\usepackage{setspace} 
%\usepackage{ctex}

%SetFonts

%SetFonts

\usepackage[inline]{asymptote}


\pagestyle{fancy}
\fancyhf{}
\rhead{Zifei (David) Zhong, B70365610 }
\lhead{\leftmark}
%\lfoot{Copyright \copyright 2021-2022 by Teacher David. All rights reserved.}

\title{Survey on Millimeter Wave Imaging}
%\author{Teacher David}
%\date{}							% Activate to display a given date or no date

\newcommand{\latex}{\LaTeX\xspace}


\begin{document}
\maketitle
%\doublespacing

%\onehalfspacing


Millimeter Wave Imaging (MWI) is a type of imaging technology that uses electromagnetic waves with wavelengths in the millimeter range to create images of objects or people. It is used in a variety of applications, including security screening, medical imaging, and non-destructive testing.

There are many ongoing studies and advancements in millimeter wave imaging technology. Some current areas of research include improving the resolution and speed of millimeter wave imaging systems, developing new imaging techniques for specific applications, and exploring the use of artificial intelligence to enhance image processing and analysis. Additionally, millimeter wave imaging is being applied to new areas, such as autonomous driving, where it can help vehicles detect and avoid obstacles in challenging weather conditions.

There are several millimeter wave imaging techniques that are used in security screening applications. One common technique is called passive millimeter wave imaging, which detects the natural millimeter wave radiation emitted by the human body and surrounding objects. This technology is often used for full-body scanners in airport security screening. Another technique is active millimeter wave imaging, which uses a millimeter wave emitter to illuminate the target and detects the reflected signal to create an image. Active millimeter wave imaging is commonly used in handheld devices for screening suspicious objects or concealed weapons. Both techniques have advantages and limitations and are used in various security screening applications depending on the specific requirements.

In addition, Millimeter wave imaging is used in medical imaging, where it has several potential applications. One such application is in breast imaging, where millimeter wave imaging can provide high-resolution images without the need for compression, which is a significant advantage over traditional mammography. Another potential application is in skin imaging, where millimeter wave imaging can provide high-resolution images of skin lesions without the need for invasive procedures. Additionally, millimeter wave imaging has been investigated for use in brain imaging, where it could potentially provide high-resolution images of brain function in real-time without the need for contrast agents. However, research in medical millimeter wave imaging is still in its early stages, and more studies are needed to fully understand its potential and limitations in various clinical applications.

Millimeter wave imaging is also used in non-destructive testing (NDT), which is a technique for inspecting materials and structures without causing damage. One of the most common applications of millimeter wave imaging in NDT is in the inspection of composite materials, which are widely used in the aerospace and automotive industries. Millimeter wave imaging can be used to detect defects and damage in composite materials that are not visible to the naked eye, such as delamination, porosity, and cracks. Another application of millimeter wave imaging in NDT is in the inspection of concrete structures, where it can be used to detect internal defects and deterioration caused by exposure to the elements or other factors. Millimeter wave imaging is also being studied for use in the inspection of metal structures and welds, where it can potentially provide high-resolution images without the need for direct contact with the material being inspected.

\end{document} 