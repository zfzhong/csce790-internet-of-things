\documentclass[11pt, oneside]{article}   	% use "amsart" instead of "article" for AMSLaTeX format
%\usepackage{geometry}                		% See geometry.pdf to learn the layout options. There are lots.
%\geometry{letterpaper}                   		% ... or a4paper or a5paper or ... 
%\geometry{landscape}                		% Activate for rotated page geometry
%\usepackage[parfill]{parskip}    		% Activate to begin paragraphs with an empty line rather than an indent

\usepackage{geometry}
 \geometry{
 a4paper,
 total={170mm,257mm},
 left=20mm,
 top=25mm,
 bottom=25mm
 }

\usepackage{graphicx}				% Use pdf, png, jpg, or eps§ with pdflatex; use eps in DVI mode
								% TeX will automatically convert eps --> pdf in pdflatex		
\usepackage{amssymb}
\usepackage{amsmath}
\usepackage{fancyhdr}
\usepackage[utf8]{inputenc}
\usepackage[english]{babel}
\usepackage{enumerate}
\usepackage{arcs}
\usepackage{cancel}
\usepackage{xfrac}
\usepackage{amsthm}
\usepackage{gensymb}
\usepackage{xspace}
\usepackage{setspace} 
%\usepackage{ctex}

%SetFonts

%SetFonts

\usepackage[inline]{asymptote}


\pagestyle{fancy}
\fancyhf{}
\rhead{Zifei (David) Zhong, B70365610 }
\lhead{\leftmark}
%\lfoot{Copyright \copyright 2021-2022 by Teacher David. All rights reserved.}

%\title{Paper title:}
%\author{Teacher David}
%\date{}							% Activate to display a given date or no date

\newcommand{\latex}{\LaTeX\xspace}


\begin{document}
%\maketitle
%\doublespacing
\onehalfspacing

\section{Paper review, due 02/02/2023}
\begin{center}
\noindent
\textbf{\Large Paper title: I am a Smartphone and I can Tell my User's Walking Direction}
\end{center}


\noindent \textbf{Student's name and ID: }\\
Zifei (David) Zhong, B70365610\\

\noindent \textbf{Summary:}\\
\emph{Briefly summarize the main ideas and the approach}.

This paper presented a system named ``WalkCompass'', which combines human walk analysis, smart phone inertial sensors, and geographic location data to estimate the walking direction of a human in environments with magnetic influence. The authors claimed that estimating a human's walking direction with the smart phone she carries, without further assumptions on starting orientation or map information, is non-trival. They described the three major components (HWA, LDE, GDE) of the WalkCompass system in details. Especially, the presented the IMT (iterative magnetic triangulation) algorithm to solve the global direction. They provided thorough evaluations on the performance of WalkCompass and did comparisons with the smart phone's original compass.  WalkCompass  They also discussed the limitations of WalkCompass.  \\

\noindent \textbf{Contributions:}\\
\emph{What are the major issues addressed in the paper?}

This paper claimed that there was no general solution to estimate the walking direction of a human based on the smart phone she carries, and it provided the WalkCompass as a generally applicable solution. \\

\noindent \emph{Do you consider them important?}

The authors showed that previous works on this topic were either based on strong assumptions (maps, initial human direction, etc) or with dissatisfactory  performance. I am convinced that it is an interesting problem and I am quite excited to see the WalkCompass solves the problem nicely. \\

\noindent \emph{Comment on the novelty, creativity, and technical depth of the paper.}

The most novel invention of this paper is the IMT (iterative magnetic triangulation) algorithm that solves the global direction. Even if I thought it's possible, the IMT algorithm just gives me `aha' moment upon reading it!

The authors presented the whole WalkCompass system nicely. I am impressed with the way they present the components of WalkCompass and explain how it works. It's a complicated system involving Human Walking analysis, inertial IMU data analysis, and iterative computation. They presented technical details with examples, and made the whole story easy to follow. Many details got presented. For example, the experiment to explain how HWA works, the analysis on walking steps, how accelerometer reflects the different stages of a walk cycle, and how measured compass vector (R) changes over distances are all details which make the whole story live.\\



\noindent \textbf{Strengths and Weaknesses:}\\
\emph{What are the paper’s strengths? Be brief.}\\

The paper provided a general solution to estimate human walking direction based on the smart phone she carries. The WalkCompass can estimate human's walking direction within about 5 steps and the performance is stable with small median errors. It is probably the first general solution that achieve this performance.\\

\noindent \emph{What are the paper's weaknesses? Be brief.}

The WalkCompass still has limitations. For example, it is unclear how it performs for the scenarios where people put smart phone in jacket and bags, and these scenarios are very common in daily life.\\

\noindent \textbf{Evaluation:}\\
\emph{Comment on the paper's evaluation methodology.}

The evaluation is quite thorough. They covered 15 different environments with 6 different users. They also provided evaluation for different special scenarios. For example, direction change, toying with the phone, walking on different surface, and even staircase and backward walking. In general, the evaluation is quite comprehensive and convincing. \\

\noindent \emph{Comment on the results.}

Given the problem of solving walking direction in environments with intense magnetic influence is a hard problem, the result of WalkCompass is excellent. Even though there are some cases WalkCompass didn't perform so well, it already demonstrated the potential of the system as a generally applicable solution. \\

\noindent \textbf{Future works:}\\
\emph{Are there any issues or directions left open?}\\
I believe the walking direction should be solvable even when the phone is put in jacket or bag. The IMT algorithm might have room to improve.\\

\noindent \emph{List a few possible extensions of this work.}

 Applying WalkCompass on people with disabilities might work as well. With better compass in nowadays smart phone, the performance of WalkCompass could be improved.\\


\noindent \textbf{Other comments or questions:}\\
Any other comments, suggestions, or questions.\\

\newpage
\section{Paper review, due 02/09/2023}
\begin{center}
\noindent
\textbf{\Large Paper title: Closing the Gaps in Inertial Motion Tracking}
\end{center}


\noindent \textbf{Student's name and ID: }\\
Zifei (David) Zhong, B70365610\\

\noindent \textbf{Summary:}\\
\emph{Briefly summarize the main ideas and the approach}.

This paper presented MUSE, a set of algorithms to track both the orientation and location of an moving object based on the IMU information of that object in 9 DoFs. With a complimentary filter design that combines Gyroscope and Magnetometer data for sensor fusion to estimate the global 3D orientation of moving objects. Experiments across a wide range of uncontrolled moving activities from both humans and things showed that MUSE could estimate global 3D orientation with low errors and outperformed the state-of-the-art methods. Based on its orientation estimation, MUSE also solved the location estimation of restricted moving activities with a particle filter design. MUSE achieved fair accuracy over a set of uncontrolled human activities subject to certain restrictions. \\

\noindent \textbf{Contributions:}\\
\emph{What are the major issues addressed in the paper?}

The major issues addressed in this paper are: 1) Estimating the global 3D orientation of moving objects with IMU data. The estimations outperformed the state-of-the-art methods; 2) Estimating the location of a IMU (attached to a moving object subject to certain movement restrictions) purely based on data from  that a single IMU.\\

\noindent \emph{Do you consider them important?}

Yes. The authors spent a great amount of effort introducing the significance of the problems and I am convinced the orientation or location tracking based on IMU is very important. Given that nowadays almost every electronic device has an IMU attached, accurate and timely tracking of location and orientation will be included by more and more applications.\\

\noindent \emph{Comment on the novelty, creativity, and technical depth of the paper.}

In the first part, this paper presented a Magnetometer + Gyroscope fusion algorithm to estimate the orientation of an object based on its IMU data. The paper provided enough technical details for readers to understand the algorithm. The proposed algorithm outperformed the state-of-the-art $A^3$ approach. The author claimed that global 3D magnetic north is a better anchor than gravity.

In the second part, this paper presented a novel particle filter approach to jointly track orientation and location. The insight that movement subject to certain restrictions provided opportunity to estimate location. The authors provided many details in designing the particle filter and how to make it computationally inexpensive. The technical details include aggregation, prediction, resampling, etc.

Overall, the technical depth is sound and the novelty in the second part of the paper is excellent.\\

\noindent \textbf{Strengths and Weaknesses:}\\
\emph{What are the paper’s strengths? Be brief.}\\

This paper has a great introduction to the ongoing research on IMU-based motion tracking, including the fundamentals of computing 3D orientation and location, the existing difficulties and challenges, and the start-of-the-art solutions. 

It provided new approaches to track orientation and location based on information from a single IMU. The approaches are either novel or practically better than existing solutions.
\\

\noindent \emph{What are the paper's weaknesses? Be brief.}

There several minor weaknesses: 1)  In evaluation of orientation estimation, the complementary filter perform badly. However, MUSE is also based on complementary filter for orientation estimation. It seems like a contradiction. 2) For evaluation of the location estimation, I would like to see how MUSE perform for other restricted motion activities other than arm movement.\\

\noindent \textbf{Evaluation:}\\
\emph{Comment on the paper's evaluation methodology.}

The evaluation experiments are properly set up. For orientation estimation, the experiments covered a very wide range of scenarios: controlled motions, uncontrolled motions, human motions, and motions of things. 

For location estimation, I would assume VICON marker-based tracking can track wrist watch as long as there is a marker on the watch. \\

\noindent \emph{Comment on the results.}

It's great to see that MUSE's magnetometer+gyroscope fusion worked better than $A^3$. Just that the bad performance of complementary filter is a bit contradict to MUSE since MUSE also uses complementary filter. Figure 17 it alone is not enough to prove that MUSE doesn't diverge error over time (To be convincing, more experiments data should be provided). \\

\noindent \textbf{Future works:}\\
\emph{Are there any issues or directions left open?}\\

As the authors mentioned in several of their papers that strong magnetic interference will pollute the magnetometer data and cause bad performance of their proposals. A study on magnetic interference and is impact might be meaningful for research in this area.\\

\noindent \emph{List a few possible extensions of this work.}

The paper only evaluated their location estimation approach for wrist-wearing watches. It would be great to extend this approach for other movements.\\


\noindent \textbf{Other comments or questions:}\\
\emph{Any other comments, suggestions, or questions}.\\
There is a typo `round' at the end of section 6.1. It should be `ground truth' in stead of `round truth'.



\newpage
\section{Paper review, due 02/23/2023}
\begin{center}
\noindent
\textbf{\Large Paper title: ``FingerIO: Using Active Sonar for Fine-Grained finger Tracking''}
\end{center}


\noindent \textbf{Student's name and ID: }\\
Zifei (David) Zhong, B70365610\\

\noindent \textbf{Summary:}\\
\emph{Briefly summarize the main ideas and the approach}.

This paper presents an approach called \emph{FingerIO} which tracks the motion of a single finger to achieve the purpose of touchless interactions with mobile devices. This approach proposes to transmit the audio signals based on the OFDM (orthogonal frequency division multiplexing) technique, in the inaudible frequency range of 18-20 KHz. Two microphones are used to locate the finger in distance, and therefore can track the motion path of the finger. The approach does not require extra devices mounted on the finger. The authors implemented FingerIO on a prototyped smart watch and a Samsung Galaxy S4 smart phone. Evaluation results show that with FingerIO humans can interact with mobile devices with good accuracy (error within ~1 centimeter). \\

\noindent \textbf{Contributions:}\\
\emph{What are the major issues addressed in the paper?}

The major issue addressed in this paper is instrumenting the microphones of consumer devices to track finger motions and hence make around-device interactions possible. The authors claim that this work does not require extra devices/sensors mounted on fingers.\\

\noindent \emph{Do you consider them important?}

Yes, I consider this work important and practical because it does not require extra instrumenting on fingers and it can be implemented on existing consumer smartphones.\\

\noindent \emph{Comment on the novelty, creativity, and technical depth of the paper.}

Comparing with existing work at the time this paper gets published (2016), the idea of this paper is quite novel because it does not require extra mounting devices for fingers. At that time, this paper probably is the first work that has made it possible.

The leveraging over OFDM is reasonable. I am not sure there are previous work that applies OFDM to audio wave transmissions. If not, this is also very creative.

This paper provided many details on the theory aspects of the wave transmission, the distance calculation, and prototype implementation. It also provided evaluation from several different aspects: accuracy, interaction surface area, effective distance, and interaction with interference.

Overall, the ideas are creative, the implementation is clean, the evaluation is systematic, and the technical depth is excellent.\\

\noindent \textbf{Strengths and Weaknesses:}\\
\emph{What are the paper’s strengths? Be brief.}\\

The strength of this paper is that it provided a solution for touchless interaction with mobile devices with a naked finger. And it is the first work that achieves this through an OFDM based ultrasonic approach. The approach is implemented on a consumer smartphone and thus it's very applicable.
\\

\noindent \emph{What are the paper's weaknesses? Be brief.}

The weaknesses, as the authors acknowledged in the paper, is that it only tracks a single finger's movement, and it seems is not able to track even cursive writing of alphabets. This indicates that the application of FingerIO might be quite limited, or it requires further development to make it more generally applicable for touchless interaction with mobile devices. \\

\noindent \textbf{Evaluation:}\\
\emph{Comment on the paper's evaluation methodology.}

The evaluation of this paper is presented in five aspects: 1) the accuracy of finger tracking; 2) the effective area of interaction surface; 3) effectiveness in occlusions; 4) false positive \& false negatives; 5) effectiveness in presence of interference.  For single-finger motion tracking, I consider their evaluation quite systematic.\\

\noindent \emph{Comment on the results.}

The proposed approach achieves great accuracy in identifying finger motions, with 0.8 cm median error for the experiment with smart phone and 1.2 cm median error for the experiment with prototyped smart watch. The largest error is about 1.4 cm for the experiment with smart phone, and I believe this demonstrates the effectiveness of the approach.

For the interaction surface experiments, their results show that the effective area of interaction surface for the phone is about $0.5$ m$^2$. While this is reasonable, I am a bit surprised to see the interaction surface area for the  prototyped watch is about $0.5 \times 0.25$ m$^2$. The author briefly explained that the difference was due to separation and power level. I believe the author should provide more details on the difference between device specifications of the prototyped smart watch and the smartphone. 

Other than that, the approach works in presence of occlusions, and in certain conditions it results false positive/negatives. I don't expect this simple finger tracking would handle complex situations. It's great the see the authors included them in the experiments.\\

\noindent \textbf{Future works:}\\
\emph{Are there any issues or directions left open?}\\

I believe this is a good work to follow for understanding how single finger motion tracking gets working with ultrasound waves. After 6 years, tracking 3D finger motions becomes the main challenge.\\

\noindent \emph{List a few possible extensions of this work.}

Tracking 3D finger motions at the joint level with ultrasound and mmWave.\\
\noindent \textbf{Other comments or questions:}\\
\emph{Any other comments, suggestions, or questions}.


\newpage
\section{Paper review, due 03/14/2023}
\begin{center}
\noindent
\textbf{\Large Paper title: ``Whole-Home Gesture Recognition Using Wireless Signals''}
\end{center}


\noindent \textbf{Student's name and ID: }\\
Zifei (David) Zhong, B70365610\\

\noindent \textbf{Summary:}\\
\emph{Briefly summarize the main ideas and the approach}.

This paper presents a system based on wireless signals to recognize human whole-body gestures in whole-home settings. The system involves a design that can identify Doppler shifts a the resolution of a few Hertz fro Wi-Fi signals. With this design, it can characterize the features of several different whole-body gestures. The system also has a mechanism to track a particular target person in an environment with multiple people presented. A proof-of-concept prototype of the system is implemented using USRP-N210s device. Evaluation from an office environment and a two-bedroom apartment indicated that the system has high accuracy over (90\%) in recognizing a set of 9 human gestures.\\

\noindent \textbf{Contributions:}\\
\emph{What are the major issues addressed in the paper?}

The major issued addressed in this paper is recognizing human body gestures in a whole-home setting with wireless signal made possible, without requiring extra devices mounted on the human body.\\


\noindent \emph{Do you consider them important?}

Yes, I consider it important. It's the first whole-home gesture recognition work that requires neither user instrumentation nor cameras.\\


\noindent \emph{Comment on the novelty, creativity, and technical depth of the paper.}

This paper presents a novel method to transform the received signals into narrowband pulse with the bandwidth of a few Hertz and then tracks the frequency of this narrowband pulse to detect the small Doppler shifts to recognize different gestures. The details of the method is provided in section 3 and that is a great piece of learning materials for other researchers. 

This paper also provide details on how it solves many practical issues in order to make the idea work successfully. With a set of 9 gestures which are carefully designed,  it provides a series of experiments to demonstrate the effectiveness of its design and implementation. This paper has shown great technical depth.\\


\noindent \textbf{Strengths and Weaknesses:}\\
\emph{What are the paper’s strengths? Be brief.}

The strength of this paper is that it has demonstrated successfully how to recognize human gestures via wireless signals without requiring on-body devices or cameras. It has designed a novel method to extract Doppler shifts from wireless signals and thus effectively characterizes different gestures. The design is implemented into a prototype and evaluated with a series of experiments with success.\\



\noindent \emph{What are the paper's weaknesses? Be brief.}

The weakness is that it only recognizes a given set of 9 gestures, which is quite a limited set of gestures. Even though it does not require on-body devices, it still requires multiple antennas in a whole-home setting. The accuracy is about 60\% (expected to be higher) when 4 receiver antennas and a single transmitter are presented.\\


\noindent \textbf{Evaluation:}\\
\emph{Comment on the paper's evaluation methodology.}

A prototype is built on the software radio platform and evaluated on the USRP-N210 hardware. The evaluation is done in two settings: an office building and a two-bedroom apartment. There are 5 users involved in  6 different experimental scenarios. 

The feasibility of gesture recognition using wireless signals in the 6 scenarios are firstly evaluated. The results indicates that using 3-4 antennas at the receiver is sufficient to achieve gesture detection in all scenarios. The gesture recognition is evaluated in the two-bedroom setting. Gesture recognition in presence of other humans is also evaluated. \\

\noindent \emph{Comment on the results.}

In gesture recognition in a two-bedroom setting, an average accuracy of 94\% is achieved with a stand deviation of 4.6\%. In kitchen or other locations where reflected signals are weak, gestures might not be detected. This non-detected gestures account for 2\% of all gestures in the experiments.

The paper also provides evaluation that indicates simply increasing the number of repetitions to 2 in the preamble will mostly decrease the false detection rate below 5\%.  More antennas will make the system more robust to interference, too.\\


\noindent \textbf{Future works:}\\
\emph{Are there any issues or directions left open?}

Certainly, this paper was published in 2013 and till now fine-grained body gesture recognition is a hot research topic and many work has been done regarding it.\\

\noindent \emph{List a few possible extensions of this work.}

Fine-grained body gesture recognition and reconstruction with mmWave technology.\\


\noindent \textbf{Other comments or questions:}\\
\emph{Any other comments, suggestions, or questions}.

\end{document} 