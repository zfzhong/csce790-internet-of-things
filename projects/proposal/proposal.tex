\documentclass[11pt, oneside]{article}   	% use "amsart" instead of "article" for AMSLaTeX format
%\usepackage{geometry}                		% See geometry.pdf to learn the layout options. There are lots.
%\geometry{letterpaper}                   		% ... or a4paper or a5paper or ... 
%\geometry{landscape}                		% Activate for rotated page geometry
%\usepackage[parfill]{parskip}    		% Activate to begin paragraphs with an empty line rather than an indent

\usepackage{geometry}
 \geometry{
 a4paper,
 total={170mm,257mm},
 left=20mm,
 top=25mm,
 bottom=25mm
 }

\usepackage{graphicx}				% Use pdf, png, jpg, or eps§ with pdflatex; use eps in DVI mode
								% TeX will automatically convert eps --> pdf in pdflatex		
\usepackage{amssymb}
\usepackage{amsmath}
\usepackage{fancyhdr}
\usepackage[utf8]{inputenc}
\usepackage[english]{babel}
\usepackage{enumerate}
\usepackage{arcs}
\usepackage{cancel}
\usepackage{xfrac}
\usepackage{amsthm}
\usepackage{gensymb}
\usepackage{xspace}
\usepackage{hyperref}
%\usepackage{ctex}

%SetFonts

%SetFonts

\usepackage[inline]{asymptote}


\pagestyle{fancy}
\fancyhf{}
%\rhead{Teacher David @ 18601688612}
\lhead{\leftmark}
%\lfoot{Copyright \copyright 2021-2022 by Teacher David. All rights reserved.}

\title{Fine-grained American Sign Language Hand Gesture Recognition Using Ultrasound and Millimeter-Wave Sensing}
\author{(Project Proposal)\\ \\Zifei (David) Zhong and Guangyi (Simona) Chen}
\date{February 6, 2022}							% Activate to display a given date or no date

\newcommand{\latex}{\LaTeX\xspace}


\begin{document}
\maketitle

\section{Introduction}
American Sign Language (ASL) hand gesture recognition is an important application of hand gesture recognition technology. Over half a million people in the United States communicate using ASL as their primary language, making ASL a vital aspect of the country's linguistic and cultural diversity. ASL hand gesture recognition technology can help bridge the communication gap between ASL users and non-ASL users, and enable more accurate and efficient translation between ASL and spoken/written language.

One of the key benefits of ASL hand gesture recognition is its ability to facilitate better communication and understanding between ASL users and non-ASL users. For example, in healthcare settings, ASL hand gesture recognition can help medical professionals communicate more effectively with patients who are deaf or hard of hearing, improving the quality of care and patient outcomes. Similarly, in educational settings, ASL hand gesture recognition can help teachers and students with hearing disabilities communicate more effectively with each other, enabling better learning outcomes and educational experiences.

ASL hand gesture recognition can also be used to develop assistive technologies for individuals with hearing disabilities. For example, ASL hand gesture recognition can be used to develop sign language translation apps or devices that can translate ASL into spoken or written language, enabling better communication and integration for ASL users in mainstream society. It can also be used in the development of smart home technology, enabling individuals with hearing disabilities to control various devices in their homes using ASL hand gestures.

In conclusion, ASL hand gesture recognition is an important application of hand gesture recognition technology that can help bridge the communication gap between ASL users and non-ASL users. It can improve communication and understanding in various settings, and can also be used to develop assistive technologies for individuals with hearing disabilities. Therefore, ASL hand gesture recognition is an important area of research and development that can have a significant impact on the lives of individuals with hearing disabilities, as well as the broader society.

Hand gesture recognition with ultrasonic technology:
\cite{ref:usgr17}, \cite{ref:usgdr21}, \cite{ref:usgr19}, \cite{ref:echoflex17}, \cite{ref:usgr18}, \cite{ref:mhgr18}, \cite{ref:hgr22}, 

There are many research work regarding ASL or other hand gesture recognition with millimeter wave technology, including \cite{ref:mmasl}, \cite{ref:espasl}, 
\cite{ref:hgrmmv21}, \cite{ref:slgrmmw22}, \cite{ref:aslmmv21}, \cite{ref:tidemo18}

Motivation: (1) ASL recognition is important, good for the community with hearing disabilities. (2) Why mobile device and mmWave technology can help? (3) Current research has limitations. (4) we propose ultrasound + mmWave, justify the reasons (goal is to improve accuracy).

\bibliographystyle{plain}
\bibliography{reference}

\end{document} 