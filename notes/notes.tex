\documentclass[11pt, oneside]{article}   	% use "amsart" instead of "article" for AMSLaTeX format
%\usepackage{geometry}                		% See geometry.pdf to learn the layout options. There are lots.
%\geometry{letterpaper}                   		% ... or a4paper or a5paper or ... 
%\geometry{landscape}                		% Activate for rotated page geometry
%\usepackage[parfill]{parskip}    		% Activate to begin paragraphs with an empty line rather than an indent

\usepackage{geometry}
 \geometry{
 a4paper,
 total={170mm,257mm},
 left=20mm,
 top=25mm,
 bottom=25mm
 }

\usepackage{graphicx}				% Use pdf, png, jpg, or eps§ with pdflatex; use eps in DVI mode
								% TeX will automatically convert eps --> pdf in pdflatex		
\usepackage{amssymb}
\usepackage{amsmath}
\usepackage{fancyhdr}
\usepackage[utf8]{inputenc}
\usepackage[english]{babel}
\usepackage{enumerate}
\usepackage{arcs}
\usepackage{cancel}
\usepackage{xfrac}
\usepackage{amsthm}
\usepackage{gensymb}
\usepackage{xspace}
%\usepackage{ctex}

%SetFonts

%SetFonts

\usepackage[inline]{asymptote}


\pagestyle{fancy}
\fancyhf{}
%\rhead{Teacher David @ 18601688612}
\lhead{\leftmark}
%\lfoot{Copyright \copyright 2021-2022 by Teacher David. All rights reserved.}

\title{Course Notes}
\author{Teacher David}
\date{January 12, 2022}							% Activate to display a given date or no date

\newcommand{\latex}{\LaTeX\xspace}


\begin{document}
\maketitle

\section{Signals}
To compare two signals, the only three key factors are: \emph{Amplitude}, \emph{Frequency}, and \emph{Phase}.

\subsection{What is a phase?}
\begin{center}
\begin{asy}
unitsize(50pt);
pair o = (0,0);
real r = 1.0;
draw(circle(o, 1));
draw(o -- dir(60)--(0.5,0), blue);
draw((-2,0)--(2,0), arrow=Arrow());
draw((0,-2)--(0,2), arrow=Arrow());
draw(arc(o, 0.3, 0, 60), red);
label("\small $\theta$", 0.4*dir(30), red);
label("\small $t$", (0.5,0), S);
\end{asy}
\end{center}

$2\pi$ angles per cycle;\\
$f$ cycles per second;\\
When time is at $t$ seconds, the angle is $\theta = 2\pi f t$, and the $\theta$ is called the \emph{Phase}.

Phase is derived from frequency: $2\pi ft$.

\textbf{We need another observer to capture both the amplitude and frequency. One from the $x$ dimension, one from the $y$ dimension.}


\subsection{Trilateration vs. Triangulation}


\end{document} 