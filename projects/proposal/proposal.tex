\documentclass[11pt, oneside]{article}   	% use "amsart" instead of "article" for AMSLaTeX format
%\usepackage{geometry}                		% See geometry.pdf to learn the layout options. There are lots.
%\geometry{letterpaper}                   		% ... or a4paper or a5paper or ... 
%\geometry{landscape}                		% Activate for rotated page geometry
%\usepackage[parfill]{parskip}    		% Activate to begin paragraphs with an empty line rather than an indent

\usepackage{geometry}
 \geometry{
 a4paper,
 total={170mm,257mm},
 left=20mm,
 top=25mm,
 bottom=25mm
 }

\usepackage{graphicx}				% Use pdf, png, jpg, or eps§ with pdflatex; use eps in DVI mode
								% TeX will automatically convert eps --> pdf in pdflatex		
\usepackage{amssymb}
\usepackage{amsmath}
\usepackage{fancyhdr}
\usepackage[utf8]{inputenc}
\usepackage[english]{babel}
\usepackage{enumerate}
\usepackage{arcs}
\usepackage{cancel}
\usepackage{xfrac}
\usepackage{amsthm}
\usepackage{gensymb}
\usepackage{xspace}
\usepackage{hyperref}
%\usepackage{ctex}

%SetFonts

%SetFonts

\usepackage[inline]{asymptote}


\pagestyle{fancy}
\fancyhf{}
\lhead{\leftmark}

\title{3D Finger Motion Tracking Using Ultrasound and Millimeter-Wave Sensing}
\author{(Project Proposal)\\ \\Zifei (David) Zhong and Guangyi (Simona) Chen}
\date{February 16, 2022}							% Activate to display a given date or no date

\newcommand{\latex}{\LaTeX\xspace}


\begin{document}
\maketitle

\section{Motivation}
In recent years, the importance and application of 3D finger motion tracking technology have expanded considerably. This technology has diverse applications in virtual reality, gaming, sign language recognition, physical therapy, and other fields. By utilizing advanced sensors and algorithms, it can accurately track the movement and position of individual fingers in three-dimensional space, leading to a wide range of innovative and practical applications.

Although motion capture cameras offer precise tracking of finger motions, they suffer from privacy concerns~\cite{ref:cameraprivacy18}, which have led researchers to explore contactless sensing techniques based on mmWave or audio signals that are privacy-preserving and can be applied to dark environments or non-line-of-sight conditions. Recent research has focused on reconstructing 3D finger motion~\cite{ref:neuropose21, ref:sslotr22, ref:mm4arm23} instead of identifying hand gestures, as an approach that can successfully reconstruct 3D finger motion can solve any hand gesture identification application.

To reconstruct 3D finger motion, many proposals rely on special devices worn by users, which limits their applications. In this project, we propose to reconstruct 3D finger motion precisely based on contactless sensing with both ultrasound and millimeter-wave techniques. We believe that the fusion of ultrasound and millimeter-wave can yield fine-grained results in reconstructing 3D finger motions.

To the best of our knowledge, this is the first work to perform continuous precise 3D finger motion tracking through the fusion of ultrasound and millimeter-wave reflections.

\section{Related Work}
In recent years, there has been a growing interest in using acoustic-based sensing and machine learning techniques for hand gesture recognition. EchoFlex~\cite{ref:echoflex17} is one such system that combines ultrasound imaging and neural network techniques to classify a set of 10 discrete hand gestures with high accuracy. Researchers have also explored other ultrasound-based approaches, such as using the Micro-Doppler effect~\cite{ref:usgr19}, range-Doppler map~\cite{ref:hgr22}, machine learning~\cite{ref:mhgr18}, image corner feature detection~\cite{ref:usgdr21}, ultrasonic rangefinder~\cite{ref:ss12}, multiple receiver antenna~\cite{ref:usgr18}, or microphone array~\cite{ref:usgr17} to classify hand gestures.  Acoustic-based sensing and applications is generally discussed in~\cite{ref:absa20}. 

There are also several studies that use millimeter wave technology for hand gesture recognition. In~\cite{ref:hgrmmv21, ref:slgrmmw22}, mmWave sensing based approaches are proposed to classify a given set of hand gestures. mmASL~\cite{ref:mmasl} is a system that uses 60 GHz millimeter-wave wireless signals to perform American Sign Language (ASL) recognition, while ExASL~\cite{ref:exasl20} recognizes sentence-level ASL. DeafSpaces~\cite{ref:deafspaces21} is another millimeter-wave-based system that can classify 15 different hand gestures. 

Recent trend in hand gesture study is 3D finger motion tracking, such as NeuroPose~\cite{ref:neuropose21}, ssLOTR~\cite{ref:sslotr22}, and mm4arm~\cite{ref:mm4arm23}. In particular, mm4arm~\cite{ref:mm4arm23} tracks track 3D finger motion with a mmWave-based approach without relying on any wearables.


Overall, these studies demonstrate the potential of using different sensing modalities and machine learning techniques to develop accurate and robust hand gesture recognition systems. However, none of them take advantages of both ultrasound and mmWave techniques to achieve joint-level fine-grain finger motion tracking.

%\section{Motivation}
%Motivation: (1) ASL recognition is important, good for the community with hearing disabilities. (2) Why mobile device and mmWave technology can help? (3) Current research has %limitations. (4) we propose ultrasound + mmWave, justify the reasons (goal is to improve accuracy).

\bibliographystyle{plain}
\bibliography{reference}

\end{document} 